\documentclass[12pt, letterpaper]{article}
\usepackage[utf8]{inputenc}
\usepackage{hyperref}

\title{Non-Associative Functions}
\author{Eric, Blake, David, Kyle, ...}
\date

\begin{document}

Pull material from here: \href{https://paper.dropbox.com/doc/Category-Theory--AssmMDYM3jAU7EX1TF7_5BWvAQ-8BBmIrxmkEh3FwKEt14J6}{Category Theory Notes}.

Direction 1: {\bf Conjecture}: Every non-associative relation (transformation?) can
be written as a nesting of multi-input associative functions.\\

Direction 2: a paper ``What {\bf isn't} composable? (And How to make it so)'' in which we lay out
some clear examples of what things {\em aren't} categories, why, and how they can be categorified.

Looks like what I've been trying to do might be \href{
  http://cglab.ca/~morin/publications/ds/bloom-submitted.pdf}{
  Operads}

\section{Non-Associative Examples}

\subsection{Exponentation $a^b$}

$$(2**3)**4 != 2**(3**4)$$

This appears to be a nice counter-example.  But its actually a trick
of the eye.  If we write as a function $f(a,b) = a^b = a**b$, we see the
problem:

\begin{equation}
(2**3)**4 = f( f(2, 3), 4) = G_4( F_2(3) )
2**(3**4) = f(2, f(3, 4)) = F_2( F_3(4) )
\end{equation}

That is not association.

This is: $f(2, f(3, f(4, x))) = F_2 \circ F_3 \circ F_4(x)$ Where
$F_n(x) := f(n, x) = n**x$ and conversely $G_x(n) := f(n,x)$.

\subsection{Cross Product $A\times B$}

A x ( B x C ) != ( A x B ) x C
The trouble with binary relations is they can have implicit symmetry breaking... in this case the right-handedness of 
(i, j, k) := î... 
sadly I wish I could tell you it was even that interesting. This is that whole "permuting the arguments and the arguments of the arguments" thing again.
W x M == det[ î, W, M]
We can thus see that the above is:
det[ î, A, det[ î, B, C]] != det[ î, det[ î, A, B], C]

Same damn thing as above! Looking at the actual functions in terms of
structure (as if you need to write a program to perform the
operations) resolves all confusion and reveals the truth.

\subsection{Octonion Multiplication}

(David)

Explicitly writing out octonion multiplication:
R-basis = {1, i1, i2, i3, i4, i5, i6, i7}
i1^2 = i2^2 = i3^2 = i4^2 = i5^2 = i6^2 = i7^2 = -1
i1*i2 = -i2*i1 = i3
i1*i3 = -i3*i1 = -i2
i1*i4 = -i4*i1 = i5
i1*i5 = -i5*i1 = -i4
i1*i6 = -i6*i1 = i7
i1*i7 = -i7*i1 = -i6
i2*i3 = -i3*i2 = i1
i2*i4 = -i4*i2 = i6
i2*i5 = -i5*i2 = i7
i2*i6 = -i6*i2 = -i4
i2*i7 = -i7*i2 = -i5
i3*i4 = -i4*i3 = i7
i3*i5 = -i5*i3 = i6
i3*i6 = -i6*i3 = -i5
i3*i7 = -i7*i3 = -i4
i4*i5 = -i5*i4 = i1
i4*i6 = -i6*i4 = i2
i4*i7 = -i7*i4 = i3
i5*i6 = -i6*i5 = i3
i5*i7 = -i7*i5 = i2
i6*i7 = -i7*i6 = i1
That should work although I'm not sure all the signs are OK

Associativity follows from the definitions of functions and function
composition. However, there are circumstances under which the clothes
worn by certain functions doesn't have to follow the associative law
-- at least, in a naive, direct sense.

Multiplication of octonions is non-associative. Identifying the
octonions with real coefficients with R^8, left multiplication by a
nonzero octonion permutes R^8 according to the composition of a rigid
motion and a scaling; if the sum of the squares of the coefficients of
this octonion is 1, then R^8 is permuted according to a rigid
motion.

Composition of rigid motions is however associative.  IIRC part of the
way these are reconciled is that the composition of two rigid motions
resulting from left-multiplications by unit octonions is not, in
general, accomplishable via a left-multiplication by a single unit
octonion. So you wind up 'stranded' in the larger group SO(8), and
IIRC any element of SO(8) can be written as the composition of 4
left-multiplications by unit octonions.

So then composition of rigid motions is associative, as it always is;
however, when some of these rigid motions wear octonion clothing, that
clothing doesn't need to look associative. I am not sure how far this
framing of associativity would extend, though.


\subsection{Path Following in $\mathbb{R}\times\mathbb{R}^2$ }

\end{document}
